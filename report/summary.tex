\documentclass[a4paper]{article}

\begin{document}
\title{Osborne 1 floppy drive emulator}
\date{July 2024}
\author{Archie Halliwell \and Lennart Leufgens \and Jun Muta \\
John Monash Science School}
\maketitle

\section{Abstract}

The Osborne 1 is a suitcase-sized portable computer released in 1981
is an important part in commputing history, being one of the first
widely available home computers. This project aims to allow the
sharing of programs for it over the internet. It sits in front of the
floppy disc module within the osborne with a small device that
interfaces with the motherboard and inserts data from a usb. This
replaces hard to find and costly 5.25'' floppy disks, which need to be
copied using legacy hardware, with usb drives, able to be loaded with
readily available files from the internet. The entire project's source
and documentation has been made readily available on github
(\url{https://github.com/archiecarrot123/osborne-floppy-emulator}) for
anyone who desires to recreate this project.

Our project came about from us trying to locate a copy of the cp/m
operating system on floppy disk for our schools osborne, for which we
did not have any software available. Upon further research we realised
locating and creating these disks prohibitively difficult for
hobbyists. We chose to create a device that can emulate a floppy disk
from a usb instead of working with floppy disks directly because of
the technical limitations of floppy disks and their mechanical
complexity. 

\subsection{Problems}

The first problem we ran into was that a capacitor in the Osborne 1
exploded when it was first plugged in---the capacitor had absorbed
moisture over the years and this caused it to heat up and
explode---but this was easily fixed by replacing the capacitor with a
modern one.

The first board we had made turned out to be too big, and collided with
the power connector on the Osborne's logic board, so we had to design a second
version, for which we used more SMD components, enabling us to shrink
it to less than half the size.

\begin{figure}
  \centering
  \includegraphics[height=5cm]{pcb-v1}
  \caption{Render of first version of PCB}
\end{figure}

\begin{figure}
  \centering
  \includegraphics[height=5cm]{pcb-v2}
  \caption{Render of second version of PCB}
\end{figure}

Once we managed to get the RP2040 to run using the clock provided by
the Osborne 1, the increased power draw caused the tiny common-mode
choke in our power supply to saturate, causing excessive power draw
and insufficient voltage. To fix this we replaced it with a hand-wound
inductor we made earlier.

\begin{figure}
  \centering
  \includegraphics[height=5cm]{inductors}
  \caption[Inductors]{SMD common-mode choke (left) and hand-wound
    coupled inductors (right)}
\end{figure}

\begin{figure}
  \centering
  \includegraphics[height=3.5cm]{pic_25_1}\hfill\includegraphics[height=3.5cm]{pic_25_4}
  \caption[Voltage and current traces with handwound inductors]{Diode
    anode voltage vs. time (yellow) and voltage across \qty{0.1}{\ohm}
    shunt (blue) with hand-wound inductors---On left under light load (in
    DCM) and on right under high load (in CCM)}
\end{figure}

\end{document}
